% Options for packages loaded elsewhere
% Options for packages loaded elsewhere
\PassOptionsToPackage{unicode}{hyperref}
\PassOptionsToPackage{hyphens}{url}
\PassOptionsToPackage{dvipsnames,svgnames,x11names}{xcolor}
%
\documentclass[
  11pt]{article}
\usepackage{xcolor}
\usepackage{amsmath,amssymb}
\setcounter{secnumdepth}{5}
\usepackage{iftex}
\ifPDFTeX
  \usepackage[T1]{fontenc}
  \usepackage[utf8]{inputenc}
  \usepackage{textcomp} % provide euro and other symbols
\else % if luatex or xetex
  \usepackage{unicode-math} % this also loads fontspec
  \defaultfontfeatures{Scale=MatchLowercase}
  \defaultfontfeatures[\rmfamily]{Ligatures=TeX,Scale=1}
\fi
\usepackage{lmodern}
\ifPDFTeX\else
  % xetex/luatex font selection
\fi
% Use upquote if available, for straight quotes in verbatim environments
\IfFileExists{upquote.sty}{\usepackage{upquote}}{}
\IfFileExists{microtype.sty}{% use microtype if available
  \usepackage[]{microtype}
  \UseMicrotypeSet[protrusion]{basicmath} % disable protrusion for tt fonts
}{}
% Make \paragraph and \subparagraph free-standing
\makeatletter
\ifx\paragraph\undefined\else
  \let\oldparagraph\paragraph
  \renewcommand{\paragraph}{
    \@ifstar
      \xxxParagraphStar
      \xxxParagraphNoStar
  }
  \newcommand{\xxxParagraphStar}[1]{\oldparagraph*{#1}\mbox{}}
  \newcommand{\xxxParagraphNoStar}[1]{\oldparagraph{#1}\mbox{}}
\fi
\ifx\subparagraph\undefined\else
  \let\oldsubparagraph\subparagraph
  \renewcommand{\subparagraph}{
    \@ifstar
      \xxxSubParagraphStar
      \xxxSubParagraphNoStar
  }
  \newcommand{\xxxSubParagraphStar}[1]{\oldsubparagraph*{#1}\mbox{}}
  \newcommand{\xxxSubParagraphNoStar}[1]{\oldsubparagraph{#1}\mbox{}}
\fi
\makeatother

\usepackage{color}
\usepackage{fancyvrb}
\newcommand{\VerbBar}{|}
\newcommand{\VERB}{\Verb[commandchars=\\\{\}]}
\DefineVerbatimEnvironment{Highlighting}{Verbatim}{commandchars=\\\{\}}
% Add ',fontsize=\small' for more characters per line
\usepackage{framed}
\definecolor{shadecolor}{RGB}{241,243,245}
\newenvironment{Shaded}{\begin{snugshade}}{\end{snugshade}}
\newcommand{\AlertTok}[1]{\textcolor[rgb]{0.68,0.00,0.00}{#1}}
\newcommand{\AnnotationTok}[1]{\textcolor[rgb]{0.37,0.37,0.37}{#1}}
\newcommand{\AttributeTok}[1]{\textcolor[rgb]{0.40,0.45,0.13}{#1}}
\newcommand{\BaseNTok}[1]{\textcolor[rgb]{0.68,0.00,0.00}{#1}}
\newcommand{\BuiltInTok}[1]{\textcolor[rgb]{0.00,0.23,0.31}{#1}}
\newcommand{\CharTok}[1]{\textcolor[rgb]{0.13,0.47,0.30}{#1}}
\newcommand{\CommentTok}[1]{\textcolor[rgb]{0.37,0.37,0.37}{#1}}
\newcommand{\CommentVarTok}[1]{\textcolor[rgb]{0.37,0.37,0.37}{\textit{#1}}}
\newcommand{\ConstantTok}[1]{\textcolor[rgb]{0.56,0.35,0.01}{#1}}
\newcommand{\ControlFlowTok}[1]{\textcolor[rgb]{0.00,0.23,0.31}{\textbf{#1}}}
\newcommand{\DataTypeTok}[1]{\textcolor[rgb]{0.68,0.00,0.00}{#1}}
\newcommand{\DecValTok}[1]{\textcolor[rgb]{0.68,0.00,0.00}{#1}}
\newcommand{\DocumentationTok}[1]{\textcolor[rgb]{0.37,0.37,0.37}{\textit{#1}}}
\newcommand{\ErrorTok}[1]{\textcolor[rgb]{0.68,0.00,0.00}{#1}}
\newcommand{\ExtensionTok}[1]{\textcolor[rgb]{0.00,0.23,0.31}{#1}}
\newcommand{\FloatTok}[1]{\textcolor[rgb]{0.68,0.00,0.00}{#1}}
\newcommand{\FunctionTok}[1]{\textcolor[rgb]{0.28,0.35,0.67}{#1}}
\newcommand{\ImportTok}[1]{\textcolor[rgb]{0.00,0.46,0.62}{#1}}
\newcommand{\InformationTok}[1]{\textcolor[rgb]{0.37,0.37,0.37}{#1}}
\newcommand{\KeywordTok}[1]{\textcolor[rgb]{0.00,0.23,0.31}{\textbf{#1}}}
\newcommand{\NormalTok}[1]{\textcolor[rgb]{0.00,0.23,0.31}{#1}}
\newcommand{\OperatorTok}[1]{\textcolor[rgb]{0.37,0.37,0.37}{#1}}
\newcommand{\OtherTok}[1]{\textcolor[rgb]{0.00,0.23,0.31}{#1}}
\newcommand{\PreprocessorTok}[1]{\textcolor[rgb]{0.68,0.00,0.00}{#1}}
\newcommand{\RegionMarkerTok}[1]{\textcolor[rgb]{0.00,0.23,0.31}{#1}}
\newcommand{\SpecialCharTok}[1]{\textcolor[rgb]{0.37,0.37,0.37}{#1}}
\newcommand{\SpecialStringTok}[1]{\textcolor[rgb]{0.13,0.47,0.30}{#1}}
\newcommand{\StringTok}[1]{\textcolor[rgb]{0.13,0.47,0.30}{#1}}
\newcommand{\VariableTok}[1]{\textcolor[rgb]{0.07,0.07,0.07}{#1}}
\newcommand{\VerbatimStringTok}[1]{\textcolor[rgb]{0.13,0.47,0.30}{#1}}
\newcommand{\WarningTok}[1]{\textcolor[rgb]{0.37,0.37,0.37}{\textit{#1}}}

\usepackage{longtable,booktabs,array}
\usepackage{calc} % for calculating minipage widths
% Correct order of tables after \paragraph or \subparagraph
\usepackage{etoolbox}
\makeatletter
\patchcmd\longtable{\par}{\if@noskipsec\mbox{}\fi\par}{}{}
\makeatother
% Allow footnotes in longtable head/foot
\IfFileExists{footnotehyper.sty}{\usepackage{footnotehyper}}{\usepackage{footnote}}
\makesavenoteenv{longtable}
\usepackage{graphicx}
\makeatletter
\newsavebox\pandoc@box
\newcommand*\pandocbounded[1]{% scales image to fit in text height/width
  \sbox\pandoc@box{#1}%
  \Gscale@div\@tempa{\textheight}{\dimexpr\ht\pandoc@box+\dp\pandoc@box\relax}%
  \Gscale@div\@tempb{\linewidth}{\wd\pandoc@box}%
  \ifdim\@tempb\p@<\@tempa\p@\let\@tempa\@tempb\fi% select the smaller of both
  \ifdim\@tempa\p@<\p@\scalebox{\@tempa}{\usebox\pandoc@box}%
  \else\usebox{\pandoc@box}%
  \fi%
}
% Set default figure placement to htbp
\def\fps@figure{htbp}
\makeatother





\setlength{\emergencystretch}{3em} % prevent overfull lines

\providecommand{\tightlist}{%
  \setlength{\itemsep}{0pt}\setlength{\parskip}{0pt}}



 
\usepackage[]{natbib}
\bibliographystyle{econ}


\usepackage{graphicx}
\usepackage{bbm}
\usepackage{amssymb}
\usepackage{amsmath}
\usepackage[nomarginpar]{geometry}
\usepackage{setspace}
\usepackage{natbib}
\usepackage{lscape}
\usepackage{rotating}
\usepackage{tabularx, calc}
\usepackage{threeparttable}
\usepackage{picinpar}  
\usepackage{longtable}
\usepackage{ae}
\usepackage{float}
\usepackage{morefloats}
\usepackage{palatino}
\usepackage{hyperref}
\usepackage{color}
\usepackage{adjustbox}
\usepackage{booktabs} 
\usepackage{ulem}
\usepackage{ragged2e}
\usepackage{changepage}
\usepackage{longtable}
%\usepackage{graphics}
%\usepackage[tableposition=top]{caption}

\geometry{left=1.0in,right=1.0in,top=1.0in,bottom=1.0in}
\onehalfspacing
\makeatletter
\@ifpackageloaded{caption}{}{\usepackage{caption}}
\AtBeginDocument{%
\ifdefined\contentsname
  \renewcommand*\contentsname{Table of contents}
\else
  \newcommand\contentsname{Table of contents}
\fi
\ifdefined\listfigurename
  \renewcommand*\listfigurename{List of Figures}
\else
  \newcommand\listfigurename{List of Figures}
\fi
\ifdefined\listtablename
  \renewcommand*\listtablename{List of Tables}
\else
  \newcommand\listtablename{List of Tables}
\fi
\ifdefined\figurename
  \renewcommand*\figurename{Figure}
\else
  \newcommand\figurename{Figure}
\fi
\ifdefined\tablename
  \renewcommand*\tablename{Table}
\else
  \newcommand\tablename{Table}
\fi
}
\@ifpackageloaded{float}{}{\usepackage{float}}
\floatstyle{ruled}
\@ifundefined{c@chapter}{\newfloat{codelisting}{h}{lop}}{\newfloat{codelisting}{h}{lop}[chapter]}
\floatname{codelisting}{Listing}
\newcommand*\listoflistings{\listof{codelisting}{List of Listings}}
\makeatother
\makeatletter
\makeatother
\makeatletter
\@ifpackageloaded{caption}{}{\usepackage{caption}}
\@ifpackageloaded{subcaption}{}{\usepackage{subcaption}}
\makeatother
\usepackage{bookmark}
\IfFileExists{xurl.sty}{\usepackage{xurl}}{} % add URL line breaks if available
\urlstyle{same}
\hypersetup{
  pdftitle={Coding assignment (QSM class)},
  pdfauthor={Hyoungchul Kim},
  pdfkeywords={3 to 6 keywords},
  colorlinks=true,
  linkcolor={cyan},
  filecolor={Maroon},
  citecolor={cyan},
  urlcolor={cyan},
  pdfcreator={LaTeX via pandoc}}


\title{Coding assignment (QSM class)}
\author{Hyoungchul Kim}
\date{September 1, 2025}
\begin{document}
\def\spacingset#1{\renewcommand{\baselinestretch}%
{#1}\small\normalsize} \spacingset{1}


%%%%%%%%%%%%%%%%%%%%%%%%%%%%%%%%%%%%%%%%%%%%%%%%%%%%%%%%%%%%%%%%%%%%%%%%%%%%%%

\date{\href{https://hchulkim.github.io}{Link to Latest version}\\ \vspace{1em}  Last updated September
1, 2025}
\title{Coding assignment (QSM class)\thanks{Some footnotes on the
title\ldots{}}}
\author{
Hyoungchul Kim\thanks{The creator. We want to thank\ldots{}}\\
The Wharton School, University of Pennsylvania\\
}
\maketitle

\bigskip
\bigskip
\begin{abstract}
This is a coding assignment for the QSM class.
\end{abstract}

\bigskip
\noindent%
{\it Keywords:} 3 to 6 keywords
\vfill

\newpage
\spacingset{1.2} % DON'T change the spacing!

\section{Use of programming
language}\label{sec-use-of-programming-language}

For questions 1-6, I used \texttt{R} programming language. For other
questions, I used \texttt{julia} programming language.

\section{Main data information}\label{sec-main-data-info}

I am using the 2022 LODES data from LEHD for my analysis. I obtained the
raw data for Philadelphia county from the LEHD website and aggregated
the data into tract-tract level. The process of cleaning the raw data
was conducted using the source code:
\texttt{src/R/01\_clean\_raw\_data.R}.

Thus my primary analysis zone will be bilateral commuting flow within
Philadelphia county for the year 2022. This means I will not be
considering the flow where either the origin or the destination is
outside of Philadelphia county.

\subsection*{Q1}\label{q1}
\addcontentsline{toc}{subsection}{Q1}

As I mentioned in Section~\ref{sec-main-data-info}, I obtained bilateral
commuting flow data from LEHD LODES for the year 2022. I also downloaded
supporting data on the locations of the tracts or blocks underlying the
the data. I downloaded them in the \texttt{input} folder. You can also
use \texttt{make\ raw} command to automatically download the raw
data.\footnote{Note that if you want to re-download the raw data, you
  must first use \texttt{make\ clean} command to remove the raw data in
  the \texttt{input} folder. Also note that if the original data was
  updated during the course of the semester, there is a likelihood that
  the new downloaded data might have different data structure that could
  affect the analysis.}

\subsection*{Q2}\label{q2}
\addcontentsline{toc}{subsection}{Q2}

For distance, I will use the distance between the centroids of the
origin and destination tracts. For this, I used \texttt{sf} package and
\texttt{tigris} package in R to calculate the distance between the
centroids of the origin and destination tracts. The source code is
available in \texttt{src/R/02\_calculate\_distance.R}. I also retained
the fixed effects estimates in the Appendix.

\subsection*{Q3}\label{q3}
\addcontentsline{toc}{subsection}{Q3}

I estimated the following linear model:

\[
\log(N_{ij}) = \theta_{i} + \lambda_{j} + \kappa d_{ij} + \varepsilon_{ij}\label{eq:distance}
\]

The estimation results are reported in the following table:

\clearpage

\begin{table}[!ht]
\centering
\caption{Estimation results}
\label{tab:est_results}

\begin{tabular}{l c}
\hline
 & Log of commuting flow \\
\hline
distance\_km          & $-0.07^{***}$ \\
                      & $(0.00)$      \\
\hline
Num. obs.             & $73326$       \\
Num. groups: w\_tract & $406$         \\
Num. groups: h\_tract & $408$         \\
R$^2$ (full model)    & $0.61$        \\
R$^2$ (proj model)    & $0.19$        \\
\hline
\multicolumn{2}{l}{\scriptsize{$^{***}p<0.001$; $^{**}p<0.01$; $^{*}p<0.05$}}
\end{tabular}

\end{table}

\footnotesize \textbf{Note}: This table presents estimates results of
the equation \ref{eq:distance}. \texttt{distance\_km} is the estimates
of the semi-elasticity of the travel time (or distance). The values in
the paranthesis in the table is the standard error.\vspace{3em}

\normalsize

You can see that the estimates of the semi-elasticity of the travel time
(or distance) is sensible as it is negative (-0.07). This indicates that
the travel time (or distance) has a negative correlation with the
commuting flow.

\section*{Q4}\label{q4}
\addcontentsline{toc}{section}{Q4}

After including all the \(ij\) pairs and adding in zero commuting flows,
I estimated the PPML model as follows:

\[
\log(\mathbb{E}[N_{ij}])= \theta_{i} + \lambda_{j} + \kappa d_{ij}\label{eq:ppml}
\]

The estimation results are reported in the following table (I also
retained the fixed effects estimates in the Appendix):

\begin{table}[!ht]
\centering
\caption{Estimation results of the PPML model}
\label{tab:est_ppml}

\begin{tabular}{l c}
\hline
 & PPML \\
\hline
distance\_km          & $-0.12^{***}$ \\
                      & $(0.01)$      \\
\hline
Num. obs.             & $165648$      \\
Num. groups: w\_tract & $406$         \\
Num. groups: h\_tract & $408$         \\
Pseudo R$^2$          & $0.64$        \\
\hline
\multicolumn{2}{l}{\scriptsize{$^{***}p<0.001$; $^{**}p<0.01$; $^{*}p<0.05$}}
\end{tabular}

\end{table}

\footnotesize \textbf{Note}: This table presents estimates results of
the equation \ref{eq:ppml}. \texttt{distance\_km} is the estimates of
the semi-elasticity of the travel time (or distance). The values in the
paranthesis in the table is the standard error.\vspace{3em}

\normalsize

We can clearly see that the etimates of the semi-elasticity of the
distance differs from the estimates in the previous question. The
absolute size of the estimates is larger in the PPML model. Intuitively,
I think this occurs because the PPML model incorporates the zero
commuting flows that were neglected in the linear model. Since the zero
commuting flows were not fully incorporated in the linear model, the
estimates in the linear model were underestimating the effect of the
travel-time cost on the commuting flow.

\clearpage

\subsection{Introduction}\label{sec-intro}

I will show some examples of things you can easily do in quarto-format.

\subsubsection{Math}\label{math}

You can easily use latex math format in quarto.

This is in-line math: \(x + y = 7\).

This is display-style math: \[x + y = 7.\]

You can also use begin align style syntax: \begin{align}
  x + y &= 7\label{eq:1}\\
  t + v &= 10.\label{eq:2}
\end{align}

This is equation \ref{eq:1}. This is equation \ref{eq:2}.

\subsubsection{Figures}\label{figures}

Putting figures is easy in quarto. Use syntax like this:

\subsection{Related literature}\label{sec-lit}

Some citation example.

\citet{schott}: The famous international trade PNTR paper.

\citet{berlin}: Another famous paper in urban/spatial economics.

You can also put them inside the paranthesis: \citep{schott, berlin}.

Also, you can cite sections: Section~\ref{sec-lit}.

\subsection{Methods}\label{sec-meth}

Some few rounds of Lorem ipsums to give you a sense of the format:

Lorem ipsum. Lorem ipsum.Lorem ipsum.Lorem ipsum.Lorem ipsum.Lorem
ipsum.Lorem ipsum.Lorem ipsum.Lorem ipsum.Lorem ipsum.Lorem ipsum.Lorem
ipsum.Lorem ipsum.Lorem ipsum.Lorem ipsum.Lorem ipsum.Lorem ipsum.Lorem
ipsum.Lorem ipsum.Lorem ipsum.Lorem ipsum.Lorem ipsum.Lorem ipsum.Lorem
ipsum.Lorem ipsum.Lorem ipsum.Lorem ipsum.Lorem ipsum.Lorem ipsum.Lorem
ipsum.Lorem ipsum.Lorem ipsum.Lorem ipsum.Lorem ipsum.Lorem ipsum.Lorem
ipsum.Lorem ipsum.Lorem ipsum.Lorem ipsum.Lorem ipsum.Lorem ipsum.Lorem
ipsum.Lorem ipsum.Lorem ipsum.Lorem ipsum.Lorem ipsum.Lorem ipsum.Lorem
ipsum.Lorem ipsum.Lorem ipsum.Lorem ipsum.Lorem ipsum.Lorem ipsum.Lorem
ipsum.Lorem ipsum.Lorem ipsum.Lorem ipsum.Lorem ipsum.Lorem ipsum.Lorem
ipsum.Lorem ipsum.Lorem ipsum.Lorem ipsum.Lorem ipsum.Lorem ipsum.Lorem
ipsum.Lorem ipsum.Lorem ipsum.Lorem ipsum.Lorem ipsum.Lorem ipsum.Lorem
ipsum.Lorem ipsum.Lorem ipsum.Lorem ipsum.Lorem ipsum.Lorem ipsum.Lorem
ipsum.Lorem ipsum.Lorem ipsum.Lorem ipsum.

Lorem ipsum.Lorem ipsum.Lorem ipsum.Lorem ipsum.Lorem ipsum.Lorem
ipsum.Lorem ipsum.Lorem ipsum.Lorem ipsum.Lorem ipsum.Lorem ipsum.Lorem
ipsum.Lorem ipsum.Lorem ipsum.Lorem ipsum.Lorem ipsum.Lorem ipsum.Lorem
ipsum.Lorem ipsum.Lorem ipsum.Lorem ipsum.Lorem ipsum.Lorem ipsum.Lorem
ipsum.Lorem ipsum.Lorem ipsum.Lorem ipsum.Lorem ipsum.Lorem ipsum.Lorem
ipsum.Lorem ipsum.Lorem ipsum.Lorem ipsum.Lorem ipsum.Lorem ipsum.Lorem
ipsum.Lorem ipsum.Lorem ipsum.Lorem ipsum.Lorem ipsum.Lorem ipsum.Lorem
ipsum.Lorem ipsum.Lorem ipsum.Lorem ipsum.Lorem ipsum.Lorem ipsum.Lorem
ipsum.Lorem ipsum.Lorem ipsum.Lorem ipsum.Lorem ipsum.Lorem ipsum.Lorem
ipsum.Lorem ipsum.Lorem ipsum.Lorem ipsum.Lorem ipsum.Lorem ipsum.Lorem
ipsum.Lorem ipsum.Lorem ipsum.Lorem ipsum.Lorem ipsum.Lorem ipsum.Lorem
ipsum.Lorem ipsum.Lorem ipsum.Lorem ipsum.Lorem ipsum.Lorem ipsum.Lorem
ipsum.Lorem ipsum.Lorem ipsum.

Lorem ipsum.Lorem ipsum.Lorem ipsum.Lorem ipsum.Lorem ipsum.Lorem
ipsum.Lorem ipsum.Lorem ipsum.Lorem ipsum.Lorem ipsum.Lorem ipsum.Lorem
ipsum.Lorem ipsum.Lorem ipsum.Lorem ipsum.Lorem ipsum.Lorem ipsum.Lorem
ipsum.Lorem ipsum.Lorem ipsum.Lorem ipsum.Lorem ipsum.Lorem ipsum.Lorem
ipsum.Lorem ipsum.Lorem ipsum.Lorem ipsum.Lorem ipsum.Lorem ipsum.Lorem
ipsum.Lorem ipsum.Lorem ipsum.Lorem ipsum.Lorem ipsum.Lorem ipsum.Lorem
ipsum.Lorem ipsum.Lorem ipsum.Lorem ipsum.Lorem ipsum.Lorem ipsum.Lorem
ipsum.Lorem ipsum.Lorem ipsum.Lorem ipsum.Lorem ipsum.Lorem ipsum.Lorem
ipsum.Lorem ipsum.Lorem ipsum.Lorem ipsum.Lorem ipsum.Lorem ipsum.Lorem
ipsum.Lorem ipsum.Lorem ipsum.Lorem ipsum.Lorem ipsum.Lorem ipsum.Lorem
ipsum.Lorem ipsum.Lorem ipsum.Lorem ipsum.Lorem ipsum.Lorem ipsum.Lorem
ipsum.Lorem ipsum.Lorem ipsum.Lorem ipsum.Lorem ipsum.Lorem ipsum.Lorem
ipsum.Lorem ipsum.Lorem ipsum.Lorem ipsum.Lorem ipsum.Lorem ipsum.Lorem
ipsum.Lorem ipsum.Lorem ipsum.Lorem ipsum.

\subsection{Results}\label{sec-result}

I am not sure you would need this, but in quarto, you can directly type
codes and get their outputs (I know it's not pretty\ldots{} You can
customize it later if you want though):

\begin{Shaded}
\begin{Highlighting}[]
\NormalTok{attitude }\SpecialCharTok{|\textgreater{}} \FunctionTok{head}\NormalTok{()}
\end{Highlighting}
\end{Shaded}

\begin{verbatim}
  rating complaints privileges learning raises critical advance
1     43         51         30       39     61       92      45
2     63         64         51       54     63       73      47
3     71         70         68       69     76       86      48
4     61         63         45       47     54       84      35
5     81         78         56       66     71       83      47
6     43         55         49       44     54       49      34
\end{verbatim}

\subsection{Conclusion}\label{sec-conc}

\subsection*{References}\label{references}
\addcontentsline{toc}{subsection}{References}

\renewcommand{\bibsection}{}
\bibliography{bibliography.bib}

\phantomsection\label{appendix}
\bigskip

\begin{center}

{\large\bf APPENDIX}

\end{center}

\texttt{\{.supplementary\}} works as an indicator that makes the section
centered. Useful when starting an appendix or supplementary sections.

\subsubsection*{Appendix A}\label{appendix-a}
\addcontentsline{toc}{subsubsection}{Appendix A}

Use \texttt{\{.unnumbered\}} to unnumber the section and give your own
numbering.

\subsubsection*{Appendix B}\label{appendix-b}
\addcontentsline{toc}{subsubsection}{Appendix B}

\phantomsection\label{supplementary}
\bigskip

\begin{center}

{\large\bf SUPPLEMENTARY}

\end{center}





\end{document}
